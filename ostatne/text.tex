\documentclass[11pt]{article}
\usepackage[slovak]{babel}
\usepackage[utf8]{inputenc}
\usepackage[T1]{fontenc}
\usepackage{geometry}
\usepackage{amsmath}
\usepackage{amsfonts}
\usepackage{amssymb}
\usepackage{mathrsfs}
\usepackage{enumitem}
\usepackage{multicol}
\usepackage{watermark}
\usepackage{framed}
\usepackage{shuffle}
\usepackage{listings}
\usepackage{xcolor}
\usepackage{graphicx}
\usepackage{hyperref}

\lstset { %
    language=C++,
    backgroundcolor=\color{black!5}, % set backgroundcolor
    basicstyle=\footnotesize,% basic font setting
}

\geometry{
  a4paper,
  top=25mm,
  bottom=18mm,
  right=21mm,
  left=22mm
}

\newcommand{\eps}{\varepsilon}

\pagestyle{plain}
\setlength{\parindent}{0pt}
\setlength{\parskip}{8pt}
\setlist{topsep=0pt}

\begin{document}

\begin{lstlisting}
    operator : operator_name '(' ')'
             | operator_name '(' parameter (',' parameter)* ')'
             ;
    
    parameter : '[' parameter (',' parameter)* ']'
              | operator
              | term
              | relation
              | string
              ;
    
    term : function_name '(' ')'
         | function_name '(' term (',' term)* ')'
         | variable
         | constant
         ;
\end{lstlisting}

Po minulej konzultácii sme došli na to, že gramatika algebry by mala vyzerať nejak takto. Je to z dôvodu, lebo v datalogu chceme podporovať predikáty ako napríklad $subtotal$, ktoré berú ďalšie predikáty ako parametre. Taktiež chceme mať možnosť podporovať ľubovoľný operátor, ktorý si uživateľ naprogramuje. Ako základné operátory algebry zoberieme klasické $JOIN$, $ANTIJOIN$, $UNION$ a $REC$ a budeme musieť ešte pridať operátor $RELATION$\footnote{Paremetre by moholi byť napríklad niekoľko listov konštát.}, ktorý vytvorí dočasnú reláciu.

Z týchto dôvodov veríme, že jednotný \textit{interface parametrov} pre operátory nie je možný
ale zároveň nie je ani potrebný. \textit{Loader} bude fungovať nasledovne. Pri spracovávaní konkrétneho operátora, dostane od ANTLeru takzvaný \href{https://stackoverflow.com/questions/24185887/understanding-the-context-data-structure-in-antlr4}{\lstinline{contex}}\footnote{https://stackoverflow.com/questions/24185887/understanding-the-context-data-structure-in-antlr4}, v ktorom bude konkréty \lstinline{operator_name} a konkrétne objekty parametrov. Pomocov týchto informácií sa rozhodne, ako (respektíve či vôbec) zavolá konkrétnu javovskú funkciu databázy.
\end{document}

