\documentclass[12pt, twoside]{book}
%\documentclass[12pt, oneside]{book}  % jednostranna tlac

\usepackage[a4paper,top=2.5cm,bottom=2.5cm,left=3.5cm,right=2cm]{geometry}
\usepackage[utf8]{inputenc}
\usepackage[T1]{fontenc}

\usepackage[slovak]{babel}

\linespread{1.25}

\usepackage{listings}
% ukazky kodu su cislovane ako Listing 1,2,...
% tu je Listing zmenene na Algoritmus 1,2,...
\renewcommand{\lstlistingname}{Algoritmus}
% nastavenia balicka listings
% mozete pridat aj language=...
% na nastavenie najcastejsie pouzivaneho prog. jazyka
% takisto sa da zapnut cislovanie riadkov
\lstset{frame=lines}

\usepackage{graphicx}
\usepackage{pdfpages}
\usepackage{url}
\usepackage[hidelinks,breaklinks]{hyperref}

\def\mfrok{2026}
\def\mfnazov{Kompilátor Datalogu s funkčnými symbolmi do relačnej algebry}
\def\mftyp{Bakalárska práca}
\def\mfautor{Matúš Duchyňa}
\def\mfskolitel{doc. Mgr. Tomáš Plachetka, Dr.}

\def\mfmiesto{Bratislava, \mfrok}

\def\mfodbor{ Informatika }
\def\program{ Informatika }

\def\mfpracovisko{ Katedra informatiky }

\begin{document}     
\frontmatter
\pagestyle{empty}

% -------------------
% --- Obalka ------
% -------------------

\begin{center}
\sc\large
Univerzita Komenského v Bratislave\\
Fakulta matematiky, fyziky a informatiky

\vfill

{\LARGE\mfnazov}\\
\mftyp
\end{center}

\vfill

{\sc\large 
\noindent \mfrok\\
\mfautor
}

\cleardoublepage
% --- koniec obalky ----

% -------------------
% --- Titulný list
% -------------------


\noindent

\begin{center}
\sc  
\large
Univerzita Komenského v Bratislave\\
Fakulta matematiky, fyziky a informatiky

\vfill

{\LARGE\mfnazov}\\
\mftyp
\end{center}

\vfill

\noindent
\begin{tabular}{ll}
Študijný program: & \program \\
Študijný odbor: & \mfodbor \\
Školiace pracovisko: & \mfpracovisko \\
Školiteľ: & \mfskolitel \\
\end{tabular}

\vfill


\noindent \mfmiesto\\
\mfautor

\cleardoublepage
% --- Koniec titulnej strany


% -------------------
% --- Zadanie z AIS
% -------------------
% v tlačenej verzii s podpismi zainteresovaných osôb.
% v elektronickej verzii sa zverejňuje zadanie bez podpisov
% v pracach v angličtine anglické aj slovenské zadanie

\newpage 
\setcounter{page}{3}
\includepdf{images/zadanie.pdf}

% --- Koniec zadania


% -------------------
%   Čestné vyhlásenie a nepovinné poďakovanie
% -------------------
\newpage 
\pagestyle{plain}
\paragraph{Čestné vyhlásenie:}
Čestne vyhlasujem, že celú bakalársku prácu na tému \uv{\mfnazov},
vrátane všetkých jej príloh a obrázkov, som vypracoval/vypracovala
samostatne, a to s použitím literatúry uvedenej v priloženom zozname.

% Uveďte na čo ste nástroje používali
% Podľa typu použitia potom nástroje aj citujte v texte
% TODO?
% Pri príprave tejto práce boli tiež použité nástroje umelej
% inteligencie [ZOZNAM NÁSTROJOV] za účelom [DÔVOD]. Nástroje umelej
% inteligencie som použil/použila v súlade s príslušnými právnymi
% predpismi, akademickými právami a slobodami, etickými a morálnymi
% zásadami za súčasného dodržania akademickej integrity. Som si
% vedomý/vedomá, že plne zodpovedám za správnosť výsledného textu.

\paragraph{Poďakovanie:} TODO % TODO

% --- Koniec poďakovania

% -------------------
%   Abstrakt - Slovensky
% -------------------
\newpage 
\section*{Abstrakt}

% TODO
Slovenský abstrakt v rozsahu 100-500 slov, jeden odstavec. Abstrakt
stručne sumarizuje výsledky práce. Mal by byť pochopiteľný pre bežného
informatika. Nemal by teda využívať skratky, termíny alebo označenie
zavedené v práci, okrem tých, ktoré sú všeobecne známe.

\paragraph*{Kľúčové slová:} datalog, relačná algebra, kompilátor, databázový systém


% -------------------
% --- Abstrakt - Anglicky 
% -------------------
\newpage 
\section*{Abstract}
% TODO
Abstract in the English language (translation of the abstract in the
Slovak language).

% TODO
\paragraph*{Keywords:} 

% --- Koniec Abstrakt - Anglicky

% -------------------
% --- Predhovor - v informatike sa zvacsa nepouziva
% -------------------
%\newpage 
%
%\chapter*{Predhovor}
%
%Predhovor je všeobecná informácia o práci, obsahuje hlavnú charakteristiku práce 
%a okolnosti jej vzniku. Autor zdôvodní výber témy, stručne informuje o cieľoch 
%a význame práce, spomenie domáci a zahraničný kontext, komu je práca určená, 
%použité metódy, stav poznania; autor stručne charakterizuje svoj prístup a svoje 
%hľadisko. 
%
% --- Koniec Predhovor


% -------------------
% --- Obsah
% -------------------


\cleardoublepage
\tableofcontents

% ---  Koniec Obsahu

% -------------------
% --- Zoznamy tabuliek, obrázkov, skratiek, slovník - nepovinné
% -------------------

\newpage 

\listoffigures
\listoftables

% ---  Koniec Zoznamov

\mainmatter
\pagestyle{headings}


\addcontentsline{toc}{chapter}{Úvod}
\chapter*{Úvod}

Táto bakalárska práca je jednou zo série prác pod vedením docenta Plachetku, ktoré sa zaoberajú experimentálnym databázovým systémom.

Klasické moderné databázové systémy poskytujú používateľom na vytváranie dotazov zväčša deklaratívne jazyky, ako napríklad SQL. Výhodou deklaratívneho jazyka, akým je SQL, je to, že používateľ sa nemusí zaoberať samotným algoritmom materializácie dotazu. Súčasťou klasických databázových modelov je tzv. \textit{optimalizér}. Tento komponent za pomoci rôznych metadát o tabuľkách a samotného SQL dotazu vytvorí \textit{vykonávací plán}, ktorý konkrétne popisuje postup databázového systému pri materializácii dotazu. Aj napriek snahe vývojárov jednotlivých systémov však optimalizér nie vždy vytvorí \textit{optimálny} vykonávací plán. Prinútiť databázový systém používať konkrétny vykonávací plán nie je väčšinou jednoduché. Preto existujú experti, ktorí sa zaoberajú výlučne touto problematikou. Klasické postupy optimalizácie zahŕňajú používanie \textit{hintov}, prepisovanie dotazu na iný ekvivalentný dotaz alebo explicitné nariadenie použitia konkrétneho plánu. Vykonávacie plány však väčšinou používajú veľké množstvo komplikovaných operácií s množstvom parametrov a bez kvalitnej dokumentácie. Kvôli tomu je písanie konkrétneho plánu neprektické a neefektívne.

Náš databázový systém\footnote{Pod \textit{naším} databázovým systémom myslím experimentálny systém, na ktorom pracuje alebo pracovali môj školiteľ a kolegovia pod jeho vedením} používa ako dotazovací jazyk relačnú algebru. Konkrétne ide o relačnú algebru s piatimi operátormi a podporou funkčných symbolov. Viac o nej je uvedené v kapitole \ref{chap:algebra}. Relačná algebra nám priamo umožňuje špecifikovať algoritmus materializácie dotazu. Pri jej návrhu sme sa zamerali na jednoduchosť, čo sa však odrazilo na veľkosti dotazov. \textit{Loader}\footnote{Kompilátor} z relačnej algebry do programovacieho jazyka Java a implementáciu jednotlivých funkcií v jazyku Java vytvorili kolegovia Biriukova \cite{RAcompilator} a Magát \cite{JAVAimple}.

Cieľom mojej bakalárskej práce je pridať datalog s funkčnými symbolmi ako ďalší dotazovací jazyk do nášho systému. Konkrétne ide o vytvorenie kompilátora z datalogu do relačnej algebry tak, aby bol použiteľný aj samostatne, bez väzby na konkrétny databázový systém.

Datalog je deklaratívny logický jazyk, ktorý umožňuje efektívne špecifikovať dotazy vrátane rekurzívnych. Na rozdiel od relačnej algebry alebo SQL sú dotazy v datalogu často stručné, čitateľné a intuitívne.

Práca je rozdelená na tri časti. V kapitole \ref{chap:datalog} bližšie popisujeme datalog ako jazyk logiky a definujeme jeho rozšírenie o funkčné symboly. Pri snahe osamostatniť náš kompilátor od nášho databázového systému sme identifikovali požiadavky na relačnú algebru. Tieto zmeny zároveň viedli k ďalším požiadavkám, pretože bolo potrebné upraviť loader. Taktiež chceme umožniť relatívne jednoduché pridávanie vlastných vstavaných funkcií v budúcnosti, čo generuje ďalšie požiadavky. Všetky zmeny\footnote{Oproti relačnej algebre definovanej v \cite{RAcompilator}} a ich odôvodnenia sú popísané v kapitole \ref{chap:algebra}. V záverečnej kapitole \ref{chap:kompilator} popisujeme samotný kompilátor, jeho jednotlivé časti vrátane gramatiky nášho datalogu a algoritmu prekladu.

\chapter{Datalog} \label{chap:datalog}
\newcommand{\inter}{\mathcal{I}}

\section{Definície}
\begin{note}
    $\true$ označuje pravdu (hodnotu \textit{true}). $\false$ označuje nepravdu (hodnotu \textit{false}). $\unknown$ označuje hodnotu \textit{unknown}.
\end{note}

Note: Definície 1.1.1. a 1.1.2. sú asi zbytočné a nezapadajú moc do zvyšku definícií.

\begin{definition}
    \textbf{$n$-árna relácia} $r$ je definovaná ako $r \subseteq D_1 \times D_2 \times \dots \times D_n$. Jednotlivé prvky relácie $r$ nazývame \textit{tuples}, \textit{záznamy} alebo \textit{usporiadané $n$-tice}. Jednotlivým zložkám $n$-tíc hovoríme \textit{atribúty} relácie. Množinám $D_1, D_2, \dots, D_n$ sa hovorí \textit{domény} (alebo \textit{typy}) atribútov relácie $r$. 
\end{definition}

\begin{definition}
    \textbf{$n$-árny predikát $p$} prislúchajúci $n$-árnej relácii $r \subseteq D_1 \times D_2 \times \dots \times D_n$ je funkcia $p: D_1 \times D_2 \times \dots \times D_n \to \{\true, \false\}$ definovaná nasledovne:
    $$p(x_1, x_2, \dots , x_n) = \true \Leftrightarrow (x_1, x_2, \dots, x_n) \in r$$
    $$p(x_1, x_2, \dots , x_n) = \false \Leftrightarrow (x_1, x_2, \dots, x_n) \not\in r$$
\end{definition}

\begin{definition}
    \label{znaky_jazyka_1}
    \textbf{Jazyk logiky} pozostáva z
    \begin{enumerate}[topsep=0pt,itemsep=-1ex,partopsep=1ex,parsep=1ex]
        \item predikátových symbolov
        \item symbolov pre premenné
        \item symbolov pre konštanty (reprezentujú prvky z domén)
        \item logických symbolov ($\land, \lnot$)
        \item symbolov všeobecných kvantifikátorov ($\forall$)
        \item symbolov pre definície ($\Rightarrow$)
        \item čiarok (\texttt{,})
        \item a zátvoriek (\texttt{(}, \texttt{)}).
    \end{enumerate}
\end{definition}

\begin{samepage}
    \begin{definition}
        Do \textbf{formúl} patria:
        \begin{enumerate}[topsep=0pt,itemsep=-1ex,partopsep=1ex,parsep=1ex]
            \item \textit{atomické formuly} $p(t_1, t_2, \dots t_n)$, kde $p$ je $n$-árny predikát a $t_1, t_2, \dots t_n$ sú konštanty alebo premenné
            \item ak $A$ a $B$ sú formuly, tak potom aj $A \land B$ a $\lnot A$ sú formuly
            \item nič iné nie je formula
        \end{enumerate}
    \end{definition}
\end{samepage}

\begin{definition}
    $A \Rightarrow B$ je \textbf{implikácia}, ak $A$ je klauzula a $B$ je atomická klauzula.
\end{definition}

\begin{definition}
    $\forall X \iota $ sa nazýva \textbf{kvantifikovaná implikácia}, kde $X$ je premenná a $\iota$ je implikácia. Ak je kvantifikovaná každá premenná v kvantifikovanej implikácií, nazývame ju \textbf{klauzula}.
\end{definition}

\begin{definition}
    $K_1 \land K_2 \land \dots \land K_n$ nazývame \textbf{teóriou}, kde $K_1, K_2, \dots, K_n$ sú klauzuly.
\end{definition}

\begin{definition}
    \label{ohodnotenie_premennych_1}
    \textbf{Ohodnotenie premenných} je funkcia, ktorá priradí každej premenne konštantu. Ohodnotenie $o$, ktoré premennej $X_i$ priradí konštantu $k_i$ (pre $i \leq n$) označujeme $o(X_1 | k_1, \dots , X_n | k_n)$. Keď vo formule $A$ resp. implikácií $\iota$ resp. teórií $K_1\land \dots \land K_n$ nahradíme všetky premenné podľa ohodnotenia $o$, výslednú formulu označujeme ako $A[o]$ resp. $\iota[o]$ resp. $K_1\land \dots \land K_n[o]$.
\end{definition}

\begin{definition}
    \textbf{Interpretácia jazyka} je funkcia, ktorá každej formule priradí jednu z hodnôt $\true, \false, \unknown$.
\end{definition}

\begin{definition}
    \label{interpretacia_1}
    Nech $\inter$ je interpretácia jazyka, $\varphi$ je formula a $o$ je ohodnotenie premenných. Potom $\inter \models_\true \varphi[o]$ resp. $\inter \models_\false \varphi[o]$ resp. $\inter \models_\unknown \varphi[o]$ označuje, že formula $\varphi[o]$ je pravdivá resp. neprevdivá resp. \textit{unknown} vzhľadom na interpretáciu jazyka $\inter$.
    
    Pravdivosť formuly definujeme nasledovne:
    \begin{enumerate}[topsep=0pt,itemsep=-1ex,partopsep=1ex,parsep=1ex]
        \item ${}$\\
        $\inter \models_\true \lnot \varphi[o]$ ak $\inter \models_\false \varphi[o]$.\\
        $\inter \models_\false \lnot \varphi[o]$ ak $\inter \models_\true \varphi[o]$.\\
        $\inter \models_\unknown \lnot \varphi[o]$ ak $\inter \models_\unknown \varphi[o]$.
        \item ${}$\\
        $\inter \models_\true \varphi[o] \land \phi[o]$ ak $\inter \models_\true \varphi[o]$ a zároveň $\inter \models_\true \phi[o]$.\\
        $\inter \models_\false \varphi[o] \land \phi[o]$ ak $\inter \models_\false \varphi[o]$ alebo $\inter \models_\false \phi[o]$.\\
        $\inter \models_\unknown \varphi[o] \land \phi[o]$ v ostatných prípadoch.
        \newpage
        \item ${}$\\
        $\inter \models_\true (\varphi \Rightarrow \phi)[o]$ ak platí
        \begin{enumerate}[topsep=0pt,itemsep=-1ex,partopsep=1ex,parsep=1ex]
            \item $\inter \models_\true \varphi[o]$ a zároveň $\inter \models_\true \phi[o]$,
            \item $\inter \models_\unknown \varphi[o]$ a zároveň $\inter \models_\true \phi[o]$,
            \item $\inter \models_\unknown \varphi[o]$ a zároveň $\inter \models_\unknown \phi[o]$,
            \item $\inter \models_\false \varphi[o]$.
        \end{enumerate}
        $\inter \models_\false (\varphi \Rightarrow \phi)[o]$ ak $\inter \models_\true \varphi[o]$ a zároveň $\inter \models_\false \phi[o]$.\\
        $\inter \models_\unknown (\varphi \Rightarrow \phi)[o]$ ak $\inter \models_\unknown \varphi[o]$ a zároveň $\inter \models_\false \phi[o]$.
        \item Pre kvantifikovaná implikácia $\forall X \iota$ platí:\\
        $\inter \models_\true \forall X \iota$ ak pre \textit{ľubovoľnú} konštantu $c$ platí $\inter \models_\true \iota[X|c]$,\\
        $\inter \models_\false \forall X \iota$ ak pre \textit{niektorú} konštantu $c$ platí $\inter \models_\false \iota[X|c]$,\\
        inak $\inter \models_\unknown \forall X \iota$.
    \end{enumerate}
\end{definition}

\begin{definition}
    Interpretáciu $\inter$ teórie $K_1 \land \dots \land K_n$ nazývame \textbf{model}, ak pre všetky $i \in \{1, \dots, n\}$ platí $\inter \models_\true K_i$.
\end{definition}

\begin{definition}
    Majme teóriu $K_1 \land \dots \land K_n$ a atomickú formulu $Q$. Potom teóriu $K_1 \land \dots \land K_n \land Q$ nazývame \textbf{teóriu s dotazom $Q$}.
\end{definition}

\begin{definition}
    Majme teóriu s dotazom $K_1 \land \dots \land K_n \land Q$, kde $X_1, \dots, X_k$ sú premenné v dotaze $Q$. \textbf{Výsledkom dotazu} (vzhľadom na model $\inter$ teórie $K_1 \land \dots \land K_n$) nazývame množinu usporiadaných $k$-tic
    $$\{(c_1, c_2, \dots, c_k) \, | \, \inter \models_\true Q[X_1|c_1, X_2|c_2, \dots , X_k|c_k]\}$$.
    
    Ak dotaz $Q$ neobsahuje žiadnu premennú a platí $\inter \models_\true Q$, výsledkom dotazu je $\true$, inak je výsledkom $\false$.
\end{definition}

\begin{note}
    Môžeme si všimnúť, že dotazom sa vieme pýtať ina otázky typu \textit{platí dotaz $p(K_1, \dots, K_n)$?} (kde $K_1, \dots, K_n$ sú konštanty). Nevieme sa pýtať na opak (\textit{neplatí...}).
    
    Týmto dokážeme odpovedať na niektoré dotazy pre teórie, ktoré nemajú model. Ako príklad môžeme zobrať teóriu
    $$\mathcal{T} = \left(\lnot p() \Rightarrow q()\right)\\
    \land\\
    \left(\lnot q() \Rightarrow p()\right)$$
    Jediná validná interpretácia $\inter_V$ je $\inter_V \models_\unknown p()$ a $\inter_V \models_\unknown q()$ (ak by sme zobrali ľubovoľnú inú interpretáciu $\inter$, došli by sme k sporu $\inter \models_\true p() \, \land \, \inter \not\models_\true p()$ resp. $\inter \models_\false p() \, \land \, \inter \not\models_\false p()$). Ak by sme sa vedeli pýtať na záporné dotazy, dostali by sme pre teórie s dotazom $\mathcal{T} \land p()$ a $\mathcal{T} \land \lnot p()$ výsledok dotazu $\false$. S našim prístupom dostaneme pre dotaz \uv{\textit{platí $p()$}} $\false$. Z čoho potom musí platiť, že $p()$ je nepravdivé.
\end{note}
\section{Rozšírenie datalogu o funkčné symboly}
V tejto kapitole rozšírime datalog z predchádzajúcej časti o funkčné symboly.
\section{Syntax rozšíreného datalogu}
Zoberme si formálny zápis relatívne jednoduchej teórie s dotazom:
\begin{align*}
    & [\forall X, Y \, (\text{parent}(X, Y) \Rightarrow \text{ancestor}(X, Y))] \, \land\\
    & [\forall X, Y, Z \, ((\text{parent}(X, Z) \land \text{ancestor}(Z, Y)) \Rightarrow \text{ancestor}(X, Y))] \, \land\\
    & [\forall X, Y \, (\text{ancestor}(X, Y))]
\end{align*}

Tento zápis sa rýchlo stáva neprehľadným a dlhým. Preto Datalog používa skrátený zápis. 

Klauzula
% $$\forall X_1, \dots, X_k [\left(p_1(t_{1, 1}, \dots, t_{1, k_1}) \land p_2(t_{2, 1}, \dots, t_{2, k_2}) \land \cdots \land p_n(t_{n, 1}, \dots, t_{n, k_n}) \right) \Rightarrow q(t_1, \dots, t_{k_q})]$$
$$\forall X_1, X_2, \dots, X_k [\left(p_1(t_{1, 1}, t_{1, 2}, \dots, t_{1, k_1}) \land  \cdots \land p_n(t_{n, 1}, t_{n, 2}, \dots, t_{n, k_n}) \right) \Rightarrow q(t_1, t_2, \dots, t_{k_q})]\text{,}$$
kde $X_1, X_2, \dots, X_k$ sú všetky premenné vyskytujúce sa v jednotlivých termoch, sa v skrátenom zápise zapíše ako
$$q(t_1, t_2, \dots, t_{k_q}) \leftarrow p_1(t_{1, 1}, t_{1, 2}, \dots, t_{1, k_1}), p_2(t_{2, 1}, t_{2, 2}, \dots, t_{2, k_2}), \dots, p_n(t_{n, 1}, t_{n, 2}, \dots, t_{n, k_n}).$$

Teórii s dotazom $K_1 \land K_2 \land \cdots \land K_n \land Q$ zodpovedá $D_1 D_2 \dots D_n ? Q$, kde $D_i$ je skrátený zápis klauzuly $K_i$.

Skorej spomenutá teória je reprezentovaná skráteným zápisom:
\begin{align*}
    \text{ancestor}(X, Y) & \leftarrow \text{parent}(X, Y).\\
    \text{ancestor}(X, Y) & \leftarrow \text{parent}(X, Z), \text{ancestor}(Z, Y).\\
    & ? \, \text{ancestor}(X, Y)
\end{align*}

\input{14wellfounded.tex}
\chapter{Relačná algebra}
\label{chap:algebra}
\section{Definície}
Ako bolo spomenuté v úvode, v tejto časti predefinujeme relačnú algebru z \cite{RAcompilator} aby spĺňala nové požiadavky.
\section{Rozšírenie algebry o funkčné symboly}
Keďže sme rozšírili datalog o funkčné symboly, musíme rozšíriť o funkčné symboly aj relačnú algebru.
\section{Syntax rozšírenej relačnej algebry}

\chapter{Kompilátor} \label{chap:kompilator}
\section{ANTLR}
Náš kompilátor používa generátor \textit{parserov} \textbf{AN}other \textbf{T}ool for \textbf{L}anguage \textbf{R}ecognition. V tejto časti si priblížime ako \textit{ANTLR} funguje a čo motivovalo zmeny v relačnej algebre.
\section{Gramatika datalogu pre ANTLR}
Program ANTLR potrebuje na vygenerovanie parseru gramatiku parsovaného jazyka.
\section{Algoritmus prekladu}
Po vygenerovaní stromu odvodenia prichádza na rad preklad do relačnej algebry.
\addcontentsline{toc}{chapter}{Záver}
\chapter*{Záver}


% -------------------
% --- Bibliografia
% -------------------


\newpage	

\backmatter

\thispagestyle{empty}
\clearpage
\addcontentsline{toc}{chapter}{Literatúra}
\bibliographystyle{plain}
\bibliography{literatura} 

%Prípadne môžete napísať literatúru priamo tu
% \begin{thebibliography}{5}
 
% \bibitem{br1} MOLINA H. G. - ULLMAN J. D. - WIDOM J., 2002, Database Systems, Upper Saddle River : Prentice-Hall, 2002, 1119 s., Pearson International edition, 0-13-098043-9

% \bibitem{br2} MOLINA H. G. - ULLMAN J. D. - WIDOM J., 2000 , Databasse System implementation, New Jersey : Prentice-Hall, 2000, 653s., ???

% \bibitem{br3} ULLMAN J. D. - WIDOM J., 1997, A First Course in Database Systems, New Jersey : Prentice-Hall, 1997, 470s., 

% \bibitem{br4} PREFUSE, 2007, The Prefuse visualization toolkit,  [online] Dostupné na internete: <http://prefuse.org/>

% \bibitem{br5} PREFUSE Forum, Sourceforge - Prefuse Forum,  [online] Dostupné na internete: <http://sourceforge.net/projects/prefuse/>

% \end{thebibliography}

%---koniec Referencii

% -------------------
%--- Prilohy---
% -------------------

%Nepovinná časť prílohy obsahuje materiály, ktoré neboli zaradené priamo  do textu. Každá príloha sa začína na novej strane.
%Zoznam príloh je súčasťou obsahu.

\end{document}
