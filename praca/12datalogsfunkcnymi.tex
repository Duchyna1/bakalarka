\section{Rozšírenie datalogu o funkčné symboly}
Doteraz sme dosádzali do predikátov iba konštanty a premenné. Rozšírime jazyk logiky o tzv. funkčné symboly. Funkčný symbol $f$ (s aritou $n$) reprezentuje ľubovoľnú funkciu $f: D_1 \times D_2 \times \dots \times D_n \to D$.

\begin{note}
    Možno doplniť niečo o tom, prečo chceme funkčné symboli.
\end{note}

\begin{definition}
    Za \textbf{term} považujeme:
    \begin{itemize}
        \item ľubovoľný 0-árny funkčný symbol (ktoré považujeme za \textit{konštanty}),
        \item ľubovoľnú premennú,
        \item ak $f$ je $n$-árny funkčný symbol a $t_1, t_2, \dots, t_n$ sú termy, potom $f(t_1, t_2, \dots, t_n)$ je term.
    \end{itemize}
    Dôležité je, že term pozostáva z konečného množstva symbolov.
\end{definition}

Rozširime definíciu jazyka logiky (\ref{znaky_jazyka_1}) o symboli označujúce funkčné symboly. Do predikátov potom nedopĺňame premenné a konštanty, ale termy.

\begin{definition}
    \textbf{Ground term} je ľubovoľný term, v ktorom sa nevyskytujú premenné.
\end{definition}

V definícii pre ohodnotenie premenných (\ref{ohodnotenie_premennych_1}) sa daná funkcia predefinuje na funkciu, ktorá priradí každej premennej nie konštantu, ale \textit{ground term}. Taktiež v definícii interpretácie jazyka (\ref{interpretacia_1}) prepíšeme 4. časť na:

\begin{quote}
    Pre kvantifikovaná implikácia $\forall X \iota$ platí:\\
    $\inter \models_\true \forall X \iota$ ak pre \textit{ľubovoľný} \textbf{ground term} $c$ platí $\inter \models_\true \iota[X|c]$,\\
    $\inter \models_\false \forall X \iota$ ak pre \textit{niektorý} \textbf{ground term} $c$ platí $\inter \models_\false \iota[X|c]$,\\
    inak $\inter \models_\unknown \forall X \iota$. 
\end{quote}

Ostatné definície môžu ostať tak, ako boli popísané v predošlej kapitole.